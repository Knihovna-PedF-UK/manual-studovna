\documentclass{article}
\usepackage{fontspec}
\setmainfont{TeX Gyre Heros}
\usepackage[czech]{babel}
\usepackage{csquotes}
\usepackage{microtype}
\usepackage{luavlna}
\usepackage[margin=2cm]{geometry}
\usepackage{responsive}
\setcounter{secnumdepth}{0} 
\usepackage{enumitem}
\setlist{nosep} %

\begin{document}
\section{Manuál pro studovnu}

\subsection{Co dělat při otevření}

\begin{enumerate}

\item Rozsvítit všechny světla (kromě koutku s věšáky, tam to nemá moc smysl)

\item Zapnout počítače na pultu, tiskárny

\item Zapnout počítače na galerii

\item Spočítat docházku za minulý den v docházkovém formuláři a nadepsat nový den

\item Srovnat židle a stoly

\item Zapnout fontánku
\end{enumerate}


\subsection{Jak zavřít}
\begin{enumerate}

\item Začněte dělat hluk, aby si čtenáři všimli, že se něco děje

\item Zavřete okna

\item Vypněte počítače
\item Nechte otevřené dveře do týmovky, ať větrá
\item Naznačte těm, kdo si zatím nevšimli, že zavíráte, že přesně to se právě děje
\item Zhasněte
\item Zamkněte
  \begin{itemize}
    \item Klíče jsou v 1. šuplíku, u pokladny
    \item Po zamčení klíče dejte na pult v půjčovně 
  \end{itemize}
\end{enumerate}


\subsection{Na co dávat pozor}

\begin{itemize}
\item  Nenechte uschnout kytky
\item  Dolévejte vodu do fontánky, když je hladina níž než žárovka
\end{itemize}

\subsection{Další úkoly}

\begin{itemize}
\item Zakládání knih po načtení prezenčního užití v Almě
\item Úklid počítačů pro studenty -- očistit klávesnice, myši, obrazovky
\item Rovnání knih
\item  Zalévání kytek
\end{itemize}

\subsection{Interakce se studenty}

\begin{itemize}
\item Každého příchozího zaznamenat do docházkového formuláře křížkem (nejdůležitějšíi!!!)
\item Veřejnost může být na počítačích jen hodinu (týká se hlavně problematických jedinců, jinak není třeba řešit)
\item Nejčastěji se ptají, kde najít knížky, časopisy, nebo něco vytisknout
\item Pokud někdo moc hlučí, většinou stačí se na ně tázavě podívat. Pokud to nepomůže,
zvedněte se a významně se na ně podívejte. Pokud ani to nepomůže, požádejte je, ať se ztiší.
\item Pokud provedou nějaký průšvih, můžete s nimi vyplnit \enquote{Zápis o porušení řádu} -- je na stole. Ale je to určené jen pro extrémní případy. 
\end{itemize}
  

\subsection{Jak najít literaturu}


\begin{itemize}
  \item Najděte regál pro druhou signaturu a podle autora se jí pokuste najít
  \item Pokud jí nemůžete najít, zkuste hledat okolo, odhadněte vhodné kandidáty podle počtu
    stránek, formátu (skripta, malá skripta, atd), popřípadě barvy obálky
  \item Zkuste se podívat na regál s literaturou k založení
\end{itemize}

\subsection{Časopisy pro katedry}
\begin{itemize}
  \item V zeleném kaslíku na polici u pultu
  \item  V deskách jsou formuláře pro jednotlivé katedry, při předání se podepisují
\end{itemize}

\subsection{Zakládání literatury}

\begin{itemize}
\item  Uživatelé by měli dávat prostudovanou literaturu na polici před pultem
\item  Literaturu je třeba načíst jako prezenční výpůjčku:
  \begin{itemize}
    \item  Knížky je možné načítat přes RFID. V pravém rohu stolu je pod deskou stolu destička, která RFID načítá.
    \item Na horní liště v Almě, vpravo od nápisu \enquote{Pedagogická fakulta\ldots}, je ikonka skládající se ze soustředných kruhů. Pokud není aktivní, klikněte na to a z nabídky čteček vyberte \enquote{Jiné}.
    \item  V Almě zvolte \enquote{Skenovat jednotky} a zaškrtněte políčko \enquote{Prezenční užití}.
    \item Ve skenování jednotek se u políčka \enquote{Skenovat čárový kód jednotky} zobrazí stejná ikonka pro RFID jako v té horní liště. Když na ní kliknete, načtou se všechny RFID kódy v dosahu RFID destičky.
    \item Načítejte knížky po jedné, při načítání víc jednotek může dojít k chybám.
    \item Načtené knížky založte.
  \end{itemize}
\item Knížky mají svoje umístění buď na štítku, nebo na prvním listu.
\item Text před lomítkem je rovný 2. signatuře, za lomítkem je umístění na regále. Řadí se podle abecedy, ale v zásadě jen podle prvního písmene .
\end{itemize}

\subsection{Tisk a kopírování}
\begin{itemize}
\item Tiskárna se zapíná tlačítkem \enquote{Power} na kontrolním panelu.
\item Můžou tisknout ze všech počítačů, normálně černobíle.
\item Barevný tisk je třeba povolit ve vlastnostech tiskárny při tisku.
\item 1 strana 2 Kč, oboustranně 3 Kč, barevná 10 Kč, ale jen pokud jsou tam velké barevné obrázky. Když je tam třeba jen barevný text, počítejte to jako černobíle. Plný ceník je přiložený na konci, nebo u tiskárny pro studenty.
\item Vybrané peníze dávejte do pokladničky a zapisujte do sběrného listu. Pokud je částka menší než 15 korun, zapište si jí stranou a do sběrného listu zapisujte až když je ta částka větší.
\item Kroužková vazba je komplikovaná, chce to trošku cviku. Pokud jste to nikdy nedělali, radši to nezkoušejte naostro.
\end{itemize}
  
\subsection{Nastavení kopírování}
\begin{itemize}
\item Při kopírování můžete nastavit spoustu možností na dotykovém displeji.
\item Na hlavní obrazovce kopírování se ujistěte, že kopie je černobílá, pokud má být a nastavte volby oboustranného tisku. 
\item Pokud kopírujete volné listy, můžete je dát do ručního podavače nahoře a okopírujou se všechny. 
\item Pro zmenšení nebo zvětšení je třeba vybrat vhodnou velikost v menu \enquote{Základ} $\rightarrow$ \enquote{Velikost sken. dok.} a zároveň v menu \enquote{Přiblížit}.
\item Pokud kopírujete víc stránek z knížky, můžete to urychlit pomocí menu \enquote{Pokročilé} $\rightarrow$ \enquote{Složený sken}. Pak nebudete muset čekat, až se předešlá stránka vytiskne, ale můžete rovnou kopírovat další stránky. 
\end{itemize}



\subsubsection{Problémy při tisku}
\begin{itemize}
  \item Pokud svítí nebo bliká oranžová kontrolka na kontrolním panelu a nic se netiskne, nastala chyba.
  \item Na displeji se píše, jaký je problém a zobrazí se popis, jak to opravit.
  \item Došel papír v zásobníku č. \ldots $\rightarrow$ zásobníky jsou šuplíky, které se otevírají zepředu, doplňte papír kde chybí.
  \item Došel toner -- volejte mě, Zuzku V., nebo někoho ze SIT.
  \item Zaseknutý papír -- podle instrukcí na displeji se podívejte na místa, která to píše -- typicky zásobníky papíru, tonery, atd. Je to hezky popsané a vykreslené, takže se toho nemusíte bát, zvládnete to.
  \end{itemize}
\subsection{Projektor}

\begin{itemize}
\item  Zapíná se ovladačem, který je ve druhém šuplíku vlevo na pultu.
\item  Jsou tam i propojovací kabely pro notebook a laserové ukazovátko.
\item  Kabely se zapínají do konektorů, které jsou u posledního okna.
\item  Může být třeba správně nastavit notebook -- ve vlastnostech zobrazení vyberte duplikování plochy.
\end{itemize}

\subsection{Studenti s handicapem}

\begin{itemize}
\item Nejen slabozrací, ale i dyslektici atd.
\item Pro slabozraké je určený počítač dole, kde mají program na čtení obrazovky. Když tak jim půjčte sluchátka.
\item Tisk zdarma
\item Všechny návštěvy studentů s handicapem pište do sešitu, který je v prostřední polici za pultem -- jméno, čas příchodu, čas odchodu,
počet vytištěných stránek (pokud tisknou)
\end{itemize}

\subsection{Kde co je}

\begin{itemize}
  \item Formuláře
    \begin{itemize}
      \item předtištěné 3. šuplík vpravo
      \item originály ve sdíleném
        \begin{itemize}
          \item Docházkový -- složka \enquote{Studovna Rett} $\rightarrow$ \enquote{studovna-dochazkovy-formular.pdf}
          \item Sběrný list -- složka \enquote{Formuláře} $\rightarrow$ \enquote{sběrné listy}
          \item Zápis o porušení pravidel
        \end{itemize}
    \end{itemize}
  \item Klíče -- 1. šuplík vpravo
    \begin{itemize}
      \item  Depozitář
      \item  Žebřík
      \item  Časopisy
      \item  Místnost se switchem
      \item  Studovna
    \end{itemize}
  \item Spotřební materiál
    \begin{itemize}
      \item Papíry
        \begin{itemize}
          \item  Jeden balík na polici u pultu
          \item  Další balíky ve switchové komůrce
        \end{itemize}
      \item  Tonery
        \begin{itemize}
          \item  V šoupací skříni za pravou kopírkou
          \item  Kroužky a desky na vazbu -- 2. police u vchodu do studovny vlevo
        \end{itemize}
    \end{itemize}
      \item Notebooky
        \begin{itemize}
      \item Pravá skříň u vchodu
    \end{itemize}
      \item Pokladna
        \begin{itemize}
          \item První šuplík vpravo
        \end{itemize}
    \end{itemize}

\end{document}



